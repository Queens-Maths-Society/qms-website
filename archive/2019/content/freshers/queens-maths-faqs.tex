\documentclass[a4paper]{article}

\usepackage{fullpage}
\usepackage[parfill]{parskip}
\usepackage{verbatim}

\title{Maths at Queens': A Student's Perspective}
\author{Jonathan Michael Foonlan Tsang (\tt{jftsang@jftsang.com})}
\date{Tuesday 23 June 2015}

\begin{document}
\maketitle

\tableofcontents

\section{Introduction}

There are certain questions that are often asked by prospective students to
Cambridge. Many of these questions are general and have been answered well by
many others. In this document, I will try to answer some questions that are
specific to maths at Queens', based on my own experiences. 

This document is \textit{incomplete}. If you would like to help me write it,
please do! 
 
\paragraph{About me} As of June 2015, I have just completed Part III at Queens',
and will be beginning a PhD here in October. I matriculated in 2011. I have been
focusing on applied mathematics, specifically fluid mechanics. 

\paragraph{Disclaimers} Please note that this document has \textit{not} been
endorsed by Queens' College, the Faculty of Mathematics, or the University of
Cambridge. If anything stated here conflicts with official information given by
these bodies, then those should be given more weight than this. My opinions may
change, and I may update or correct anything I say below without warning. 

\section{Overview of Cambridge}

\subsection{The collegiate system}

The University of Cambridge is a \textit{collegiate} university. It consists of
31 \textit{colleges}, as well as a number of \textit{faculties}. Each student at
Cambridge is a member of a college. The faculties are responsible for organising
lectures and setting exams. The colleges are responsible for organising
\textit{supervisions} for its students, and for their accommodation and welfare,
and other non-academic parts of student life. The central university
administration is relatively small and does not have much direct interaction
with students.

\subsection{Supervisions}


\section{Frequently Asked Questions}

\subsection{Should I apply for maths?} 

For most people who are considering this question, the answer is `yes'.

You can get a better idea of how to answer this question by looking at the
course. Information about the course, including an informal description of each
lecture course's content, is available at
\texttt{http://maths.cam.ac.uk/undergrad/course/}. 

You will notice that there is a broad choice of lecture courses, and that by
third year you can be fairly specialised. In particular, there are many courses
on theoretical physics. If you are trying to decide between the Maths course and
the Natural Sciences course, with the eventual aim of studying theoretical
physics, then the Maths course may be a better foundation. 

Most maths graduates go on to further study, such as Part III, and research.
Mathematicians can use their training to contribute towards research in areas as
diverse as plant sciences, medicine, economics and oceanography. Many other
graduates go on to take jobs in banking and IT.  There are plenty of
opportunities for summer projects or internships, which can give you a taster of
research or working.

\subsection{What makes maths at Cambridge special?}

The main difference between the Cambridge Maths Tripos and most other maths
courses is that the Tripos covers much more content. It will better prepare you
for a graduate course or research in maths. 

This also means that the Cambridge course is more fast-paced. But \textit{don't
despair} -- there's also a lot more support. You will get plenty of contact time
and feedback from your supervisors. With the collegiate system, you will be a
member of a close-knit community, and can always rely on somebody for support,
both academic and pastoral. 

\subsection{How do I choose a college? Are there any colleges that are better
than others?}

Every college has its unique properties, and it's impossible to say that one
college is objectively `better' than another. You could consider a number of
things:

\begin{itemize}
\item How large is the college? 
\item How many students are there?
\item Where is the college?
\item Is it close to town? To lectures? To the CMS? 
\item Does the college provide on-site accommodation? For all three years? Is there an option to live off-site? Which do I prefer?
\item What are the rooms like? Do I like their style? 
\item How are the rooms priced? Are they good value for money?
\item How are rooms allocated?
\item What sort of bursaries can the college award?
\item Does the college have a maths society? A music society? A drama society? A debating society? A chess club?
\item Does it have a gym? Sports grounds? 
\item Does the college encourage or discourage gap years?
\item How many mathmos does it take each year?
\item How many teaching staff does it have for maths?
\item What sort of support is given to incoming mathmos?
\end{itemize}

Most of these are general questions about college life and are relevant
regardless of what subject you study. A lot of them are subjective, and will
depend on your personal tastes.

I would advise against naively comparing numbers across colleges. You should not
choose a college based on statistics of the form `There are $r$ applicants for
every place at \dots'. 

\subsection{What makes maths at Queens' special?}

Although `which college is best?' is a subjective question, I can offer some
reasons why I've enjoyed maths at Queens'. 

\paragraph{Size} Queens' takes in 14-18 mathmos each year. This is the second
largest cohort, after Trinity. This is a healthy size: it is small enough that
you can know everyone else, but large enough that you have other people with
whom you can work. Your supervision partner is likely also to be a Queens'
student, which is useful as it means you can confer before, during and after
supervisions easily. 

\paragraph{Examples classes} Maths at Queens' is one of very few subject-college
combinations that offer extra support in the form of examples classes, in
addition to lectures and supervisions. During examples classes, a supervisor
will go through questions which everybody had problems with. This gives you more
contact time, an opportunity to get to know the other mathmos, and the
reassurance that everybody else is struggling just as you are.

\paragraph{Gap years} Most colleges discourage people who want to apply for
maths from taking gap years, and discriminate against applicants who have.
Queens' does not. 

\paragraph{College life} Queens' is a large college and has many societies. We
have our own Queens' Mathematics Society, which puts on talks and social
activities, as well as the annual maths dinner. We also have a very large music
society. 

Queens' offers on-site accommodation to everybody for three years. Other
colleges offer college-owned accommodation, but this is usually off-site and
nowhere near as central.

\subsection{Do they read the personal statement?}

Yes.

\subsection{How should I write my personal statement?}

Because maths at school is very different from maths at university, you should
show that you have some understanding of what the latter is about. You also need
to show that you are interested in maths beyond the narrow confines of the
A-level course. A good way to do this is to mention some books that you might
have read, be they books for a popular audience or textbooks for freshers (but
make sure you've actually read the book before your interview!). 

\subsection{What's the interview like?}

The interview process for maths varies between different colleges. At Queens',
you will have two interviews. Each interview will be around 20 minutes long, and
you sit at a desk with two teaching staff who will ask you \textit{maths}
questions. (You won't be asked the sort of esoteric questions that the
\textit{Daily Mail} thinks appear in Cambridge interviews.)

Their aim is not to test your ability at A-level material, as they will have
your AS results by then. Nor is it to test your accuracy with calculations and
algebraic manipulation, as they understand that you may be feeling anxious or
stressed. Nor is it to test your knowledge of university maths and
beyond---although showing such knowledge may indicate enthusiasm, you are not
expected to know any. (There is no need to try to impress the interviewer by
looking up their research interests beforehand and trying to talk about that.
Indeed, doing so may come across as pretentious.) 

The questions that they ask you will not require techniques beyond those which
you already know well, but they will be more complicated and require more
thought than A-level questions. 

\subsection{What grades do I need to get in?}

The standard offer is A*A*A at A-level, including Maths and \textit{at least} AS
Further Maths, as well as STEP. If your school offers A2 Further Maths, then you
will need this as well. If you already have a strong Maths A-level when you
apply, your conditional offer may \textit{exclude} this. 

If you are taking a full Further Maths A-level, you will also be asked to sit
STEPs 2 and 3. If you are taking only Further Maths AS, you will be asked to sit
STEPs 1 and 2. The standard offer is a Grade 1 in both papers. 

Other than the Core and Further Pure modules, there is no requirement on the
modules that you take, but Mechanics modules are preferred to Statistics and
Decision modules, as they will prepare you for a similar course (`Dynamics and
Relativity') in first year. Don't worry if you can't take all of these modules.
There is an optional course in the first term of first year for those who
haven't done Mechanics up to M3. Statistics modules are useful if you want to
attempt the Statistics questions on STEP, but no knowledge of them will be
assumed in the Tripos.

Cambridge is unusual in that it will ask for the individual module
\textit{scores} (not just grades) that you got at AS-level. 

\subsection{How hard is STEP?}

This is hard to answer objectively. A STEP paper has 13 questions on it, and you
are awarded marks for your best 6 attempts. Each question is marked out of 20,
so the total mark for a paper is 120. However, the grade boundaries vary
greatly, as STEP varies in difficulty from year to year, and a harder year is
often balanced by a lower grade boundary. A common \textit{guideline} is that
answering 4 questions well corresponds to a Grade 1.

\subsection{How can I prepare for STEP?}

Do lots of past papers. Get help from your school, if any teachers are
available. Look at the STEP resources
(available at \texttt{http://maths.cam.ac.uk/undergrad/admissions/step/}). In
particular, Stephen Siklos has written a book that works through many STEP
problems, giving hints and analysis on the way.  You might be able to sign up
for the STEP Correspondence Course (\texttt{https://correspondence.maths.org}).


\end{document}
