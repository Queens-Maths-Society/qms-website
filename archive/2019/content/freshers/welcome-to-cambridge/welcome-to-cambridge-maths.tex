\documentclass[a4paper,11pt]{article}
% Jonny's standard preamble. 
% To include this preamble on a document, insert the following line after the \documentclass line:
% % Jonny's standard preamble. 
% To include this preamble on a document, insert the following line after the \documentclass line:
% % Jonny's standard preamble. 
% To include this preamble on a document, insert the following line after the \documentclass line:
% \input{preamble.tex}

%\documentclass[a4paper,11pt,twocolumn]{article}
%\input{preamble.tex}


\usepackage{amsmath,amsthm,amssymb,amsfonts}

\usepackage[cm]{fullpage}
%\usepackage[a4page]{geometry}
%\usepackage[parfill]{parskip}
\usepackage{parskip}
\usepackage{url}
\usepackage{hyperref}
\usepackage{bm} % Nicer bolds for vectors (including bold Greek letters and bold grad symbol)
\usepackage{verbatim} % Include files verbatim using \verbatiminput

\usepackage{graphicx}
\usepackage{rotating}
\usepackage{subfigure}
% TikZ: A package for basic line drawings:
% https://www.sharelatex.com/blog/2013/08/27/tikz-series-pt1.html
\usepackage{tikz}
\usetikzlibrary{arrows,shapes,shapes.geometric,shapes.misc}
\usepackage{pgfplots}

% commath: common maths things
% http://anorien.csc.warwick.ac.uk/mirrors/CTAN/macros/latex/contrib/commath/commath.pdf 
\usepackage{commath}

\usepackage[final]{pdfpages}
\usepackage[autostyle]{csquotes}

% Some shortcuts for vector calculus. 
\newcommand{\bs}{\boldsymbol}
\newcommand{\grad}{\bs{\nabla}}
\newcommand{\cross}{\times}
\newcommand{\curl}{\grad\cross}
\newcommand{\divg}{\grad\cdot}
\newcommand{\DDtfull}{\left(\frac{\partial}{\partial t} + \bs{u}\cdot\grad\right)}
\newcommand{\DDt} [1] {\frac{\mathrm{D}#1}{\mathrm{D}t}}
\newcommand{\dDDt} [1] {\displaystyle\frac{\mathrm{D}#1}{\mathrm{D}t}} 

% Asterisks: \ask
%%% See http://tex.stackexchange.com/questions/59183/normal-high-asterisk-in-equation-mode
%%% magic code starts
\mathcode`*=\string"8000
\begingroup
\catcode`*=\active
\xdef*{\noexpand\textup{\string*}}
\endgroup
%%% magic code ends

% Shortcuts for some mathematical functions
\newcommand{\sgn}{\mathrm{sgn}}
\newcommand{\erf}{\mathrm{erf}}
\newcommand{\erfc}{\mathrm{erfc}}
\newcommand \reals {\mathbb{R}}
\newcommand \complex {\mathbb{C}}
\newcommand \Rey {\mathrm{Re}}

%%% Column vectors: see http://tex.stackexchange.com/questions/2705/typesetting-column-vector
\newcount\colveccount
\newcommand*\colvec[1]{
        \global\colveccount#1
        \begin{pmatrix}
        \colvecnext
}
\def\colvecnext#1{
        #1
        \global\advance\colveccount-1
        \ifnum\colveccount>0
                \\
                \expandafter\colvecnext
        \else
                \end{pmatrix}
        \fi
}


%\documentclass[a4paper,11pt,twocolumn]{article}
%% Jonny's standard preamble. 
% To include this preamble on a document, insert the following line after the \documentclass line:
% \input{preamble.tex}

%\documentclass[a4paper,11pt,twocolumn]{article}
%\input{preamble.tex}


\usepackage{amsmath,amsthm,amssymb,amsfonts}

\usepackage[cm]{fullpage}
%\usepackage[a4page]{geometry}
%\usepackage[parfill]{parskip}
\usepackage{parskip}
\usepackage{url}
\usepackage{hyperref}
\usepackage{bm} % Nicer bolds for vectors (including bold Greek letters and bold grad symbol)
\usepackage{verbatim} % Include files verbatim using \verbatiminput

\usepackage{graphicx}
\usepackage{rotating}
\usepackage{subfigure}
% TikZ: A package for basic line drawings:
% https://www.sharelatex.com/blog/2013/08/27/tikz-series-pt1.html
\usepackage{tikz}
\usetikzlibrary{arrows,shapes,shapes.geometric,shapes.misc}
\usepackage{pgfplots}

% commath: common maths things
% http://anorien.csc.warwick.ac.uk/mirrors/CTAN/macros/latex/contrib/commath/commath.pdf 
\usepackage{commath}

\usepackage[final]{pdfpages}
\usepackage[autostyle]{csquotes}

% Some shortcuts for vector calculus. 
\newcommand{\bs}{\boldsymbol}
\newcommand{\grad}{\bs{\nabla}}
\newcommand{\cross}{\times}
\newcommand{\curl}{\grad\cross}
\newcommand{\divg}{\grad\cdot}
\newcommand{\DDtfull}{\left(\frac{\partial}{\partial t} + \bs{u}\cdot\grad\right)}
\newcommand{\DDt} [1] {\frac{\mathrm{D}#1}{\mathrm{D}t}}
\newcommand{\dDDt} [1] {\displaystyle\frac{\mathrm{D}#1}{\mathrm{D}t}} 

% Asterisks: \ask
%%% See http://tex.stackexchange.com/questions/59183/normal-high-asterisk-in-equation-mode
%%% magic code starts
\mathcode`*=\string"8000
\begingroup
\catcode`*=\active
\xdef*{\noexpand\textup{\string*}}
\endgroup
%%% magic code ends

% Shortcuts for some mathematical functions
\newcommand{\sgn}{\mathrm{sgn}}
\newcommand{\erf}{\mathrm{erf}}
\newcommand{\erfc}{\mathrm{erfc}}
\newcommand \reals {\mathbb{R}}
\newcommand \complex {\mathbb{C}}
\newcommand \Rey {\mathrm{Re}}

%%% Column vectors: see http://tex.stackexchange.com/questions/2705/typesetting-column-vector
\newcount\colveccount
\newcommand*\colvec[1]{
        \global\colveccount#1
        \begin{pmatrix}
        \colvecnext
}
\def\colvecnext#1{
        #1
        \global\advance\colveccount-1
        \ifnum\colveccount>0
                \\
                \expandafter\colvecnext
        \else
                \end{pmatrix}
        \fi
}



\usepackage{amsmath,amsthm,amssymb,amsfonts}

\usepackage[cm]{fullpage}
%\usepackage[a4page]{geometry}
%\usepackage[parfill]{parskip}
\usepackage{parskip}
\usepackage{url}
\usepackage{hyperref}
\usepackage{bm} % Nicer bolds for vectors (including bold Greek letters and bold grad symbol)
\usepackage{verbatim} % Include files verbatim using \verbatiminput

\usepackage{graphicx}
\usepackage{rotating}
\usepackage{subfigure}
% TikZ: A package for basic line drawings:
% https://www.sharelatex.com/blog/2013/08/27/tikz-series-pt1.html
\usepackage{tikz}
\usetikzlibrary{arrows,shapes,shapes.geometric,shapes.misc}
\usepackage{pgfplots}

% commath: common maths things
% http://anorien.csc.warwick.ac.uk/mirrors/CTAN/macros/latex/contrib/commath/commath.pdf 
\usepackage{commath}

\usepackage[final]{pdfpages}
\usepackage[autostyle]{csquotes}

% Some shortcuts for vector calculus. 
\newcommand{\bs}{\boldsymbol}
\newcommand{\grad}{\bs{\nabla}}
\newcommand{\cross}{\times}
\newcommand{\curl}{\grad\cross}
\newcommand{\divg}{\grad\cdot}
\newcommand{\DDtfull}{\left(\frac{\partial}{\partial t} + \bs{u}\cdot\grad\right)}
\newcommand{\DDt} [1] {\frac{\mathrm{D}#1}{\mathrm{D}t}}
\newcommand{\dDDt} [1] {\displaystyle\frac{\mathrm{D}#1}{\mathrm{D}t}} 

% Asterisks: \ask
%%% See http://tex.stackexchange.com/questions/59183/normal-high-asterisk-in-equation-mode
%%% magic code starts
\mathcode`*=\string"8000
\begingroup
\catcode`*=\active
\xdef*{\noexpand\textup{\string*}}
\endgroup
%%% magic code ends

% Shortcuts for some mathematical functions
\newcommand{\sgn}{\mathrm{sgn}}
\newcommand{\erf}{\mathrm{erf}}
\newcommand{\erfc}{\mathrm{erfc}}
\newcommand \reals {\mathbb{R}}
\newcommand \complex {\mathbb{C}}
\newcommand \Rey {\mathrm{Re}}

%%% Column vectors: see http://tex.stackexchange.com/questions/2705/typesetting-column-vector
\newcount\colveccount
\newcommand*\colvec[1]{
        \global\colveccount#1
        \begin{pmatrix}
        \colvecnext
}
\def\colvecnext#1{
        #1
        \global\advance\colveccount-1
        \ifnum\colveccount>0
                \\
                \expandafter\colvecnext
        \else
                \end{pmatrix}
        \fi
}


%\documentclass[a4paper,11pt,twocolumn]{article}
%% Jonny's standard preamble. 
% To include this preamble on a document, insert the following line after the \documentclass line:
% % Jonny's standard preamble. 
% To include this preamble on a document, insert the following line after the \documentclass line:
% \input{preamble.tex}

%\documentclass[a4paper,11pt,twocolumn]{article}
%\input{preamble.tex}


\usepackage{amsmath,amsthm,amssymb,amsfonts}

\usepackage[cm]{fullpage}
%\usepackage[a4page]{geometry}
%\usepackage[parfill]{parskip}
\usepackage{parskip}
\usepackage{url}
\usepackage{hyperref}
\usepackage{bm} % Nicer bolds for vectors (including bold Greek letters and bold grad symbol)
\usepackage{verbatim} % Include files verbatim using \verbatiminput

\usepackage{graphicx}
\usepackage{rotating}
\usepackage{subfigure}
% TikZ: A package for basic line drawings:
% https://www.sharelatex.com/blog/2013/08/27/tikz-series-pt1.html
\usepackage{tikz}
\usetikzlibrary{arrows,shapes,shapes.geometric,shapes.misc}
\usepackage{pgfplots}

% commath: common maths things
% http://anorien.csc.warwick.ac.uk/mirrors/CTAN/macros/latex/contrib/commath/commath.pdf 
\usepackage{commath}

\usepackage[final]{pdfpages}
\usepackage[autostyle]{csquotes}

% Some shortcuts for vector calculus. 
\newcommand{\bs}{\boldsymbol}
\newcommand{\grad}{\bs{\nabla}}
\newcommand{\cross}{\times}
\newcommand{\curl}{\grad\cross}
\newcommand{\divg}{\grad\cdot}
\newcommand{\DDtfull}{\left(\frac{\partial}{\partial t} + \bs{u}\cdot\grad\right)}
\newcommand{\DDt} [1] {\frac{\mathrm{D}#1}{\mathrm{D}t}}
\newcommand{\dDDt} [1] {\displaystyle\frac{\mathrm{D}#1}{\mathrm{D}t}} 

% Asterisks: \ask
%%% See http://tex.stackexchange.com/questions/59183/normal-high-asterisk-in-equation-mode
%%% magic code starts
\mathcode`*=\string"8000
\begingroup
\catcode`*=\active
\xdef*{\noexpand\textup{\string*}}
\endgroup
%%% magic code ends

% Shortcuts for some mathematical functions
\newcommand{\sgn}{\mathrm{sgn}}
\newcommand{\erf}{\mathrm{erf}}
\newcommand{\erfc}{\mathrm{erfc}}
\newcommand \reals {\mathbb{R}}
\newcommand \complex {\mathbb{C}}
\newcommand \Rey {\mathrm{Re}}

%%% Column vectors: see http://tex.stackexchange.com/questions/2705/typesetting-column-vector
\newcount\colveccount
\newcommand*\colvec[1]{
        \global\colveccount#1
        \begin{pmatrix}
        \colvecnext
}
\def\colvecnext#1{
        #1
        \global\advance\colveccount-1
        \ifnum\colveccount>0
                \\
                \expandafter\colvecnext
        \else
                \end{pmatrix}
        \fi
}


%\documentclass[a4paper,11pt,twocolumn]{article}
%% Jonny's standard preamble. 
% To include this preamble on a document, insert the following line after the \documentclass line:
% \input{preamble.tex}

%\documentclass[a4paper,11pt,twocolumn]{article}
%\input{preamble.tex}


\usepackage{amsmath,amsthm,amssymb,amsfonts}

\usepackage[cm]{fullpage}
%\usepackage[a4page]{geometry}
%\usepackage[parfill]{parskip}
\usepackage{parskip}
\usepackage{url}
\usepackage{hyperref}
\usepackage{bm} % Nicer bolds for vectors (including bold Greek letters and bold grad symbol)
\usepackage{verbatim} % Include files verbatim using \verbatiminput

\usepackage{graphicx}
\usepackage{rotating}
\usepackage{subfigure}
% TikZ: A package for basic line drawings:
% https://www.sharelatex.com/blog/2013/08/27/tikz-series-pt1.html
\usepackage{tikz}
\usetikzlibrary{arrows,shapes,shapes.geometric,shapes.misc}
\usepackage{pgfplots}

% commath: common maths things
% http://anorien.csc.warwick.ac.uk/mirrors/CTAN/macros/latex/contrib/commath/commath.pdf 
\usepackage{commath}

\usepackage[final]{pdfpages}
\usepackage[autostyle]{csquotes}

% Some shortcuts for vector calculus. 
\newcommand{\bs}{\boldsymbol}
\newcommand{\grad}{\bs{\nabla}}
\newcommand{\cross}{\times}
\newcommand{\curl}{\grad\cross}
\newcommand{\divg}{\grad\cdot}
\newcommand{\DDtfull}{\left(\frac{\partial}{\partial t} + \bs{u}\cdot\grad\right)}
\newcommand{\DDt} [1] {\frac{\mathrm{D}#1}{\mathrm{D}t}}
\newcommand{\dDDt} [1] {\displaystyle\frac{\mathrm{D}#1}{\mathrm{D}t}} 

% Asterisks: \ask
%%% See http://tex.stackexchange.com/questions/59183/normal-high-asterisk-in-equation-mode
%%% magic code starts
\mathcode`*=\string"8000
\begingroup
\catcode`*=\active
\xdef*{\noexpand\textup{\string*}}
\endgroup
%%% magic code ends

% Shortcuts for some mathematical functions
\newcommand{\sgn}{\mathrm{sgn}}
\newcommand{\erf}{\mathrm{erf}}
\newcommand{\erfc}{\mathrm{erfc}}
\newcommand \reals {\mathbb{R}}
\newcommand \complex {\mathbb{C}}
\newcommand \Rey {\mathrm{Re}}

%%% Column vectors: see http://tex.stackexchange.com/questions/2705/typesetting-column-vector
\newcount\colveccount
\newcommand*\colvec[1]{
        \global\colveccount#1
        \begin{pmatrix}
        \colvecnext
}
\def\colvecnext#1{
        #1
        \global\advance\colveccount-1
        \ifnum\colveccount>0
                \\
                \expandafter\colvecnext
        \else
                \end{pmatrix}
        \fi
}



\usepackage{amsmath,amsthm,amssymb,amsfonts}

\usepackage[cm]{fullpage}
%\usepackage[a4page]{geometry}
%\usepackage[parfill]{parskip}
\usepackage{parskip}
\usepackage{url}
\usepackage{hyperref}
\usepackage{bm} % Nicer bolds for vectors (including bold Greek letters and bold grad symbol)
\usepackage{verbatim} % Include files verbatim using \verbatiminput

\usepackage{graphicx}
\usepackage{rotating}
\usepackage{subfigure}
% TikZ: A package for basic line drawings:
% https://www.sharelatex.com/blog/2013/08/27/tikz-series-pt1.html
\usepackage{tikz}
\usetikzlibrary{arrows,shapes,shapes.geometric,shapes.misc}
\usepackage{pgfplots}

% commath: common maths things
% http://anorien.csc.warwick.ac.uk/mirrors/CTAN/macros/latex/contrib/commath/commath.pdf 
\usepackage{commath}

\usepackage[final]{pdfpages}
\usepackage[autostyle]{csquotes}

% Some shortcuts for vector calculus. 
\newcommand{\bs}{\boldsymbol}
\newcommand{\grad}{\bs{\nabla}}
\newcommand{\cross}{\times}
\newcommand{\curl}{\grad\cross}
\newcommand{\divg}{\grad\cdot}
\newcommand{\DDtfull}{\left(\frac{\partial}{\partial t} + \bs{u}\cdot\grad\right)}
\newcommand{\DDt} [1] {\frac{\mathrm{D}#1}{\mathrm{D}t}}
\newcommand{\dDDt} [1] {\displaystyle\frac{\mathrm{D}#1}{\mathrm{D}t}} 

% Asterisks: \ask
%%% See http://tex.stackexchange.com/questions/59183/normal-high-asterisk-in-equation-mode
%%% magic code starts
\mathcode`*=\string"8000
\begingroup
\catcode`*=\active
\xdef*{\noexpand\textup{\string*}}
\endgroup
%%% magic code ends

% Shortcuts for some mathematical functions
\newcommand{\sgn}{\mathrm{sgn}}
\newcommand{\erf}{\mathrm{erf}}
\newcommand{\erfc}{\mathrm{erfc}}
\newcommand \reals {\mathbb{R}}
\newcommand \complex {\mathbb{C}}
\newcommand \Rey {\mathrm{Re}}

%%% Column vectors: see http://tex.stackexchange.com/questions/2705/typesetting-column-vector
\newcount\colveccount
\newcommand*\colvec[1]{
        \global\colveccount#1
        \begin{pmatrix}
        \colvecnext
}
\def\colvecnext#1{
        #1
        \global\advance\colveccount-1
        \ifnum\colveccount>0
                \\
                \expandafter\colvecnext
        \else
                \end{pmatrix}
        \fi
}



\usepackage{amsmath,amsthm,amssymb,amsfonts}

\usepackage[cm]{fullpage}
%\usepackage[a4page]{geometry}
%\usepackage[parfill]{parskip}
\usepackage{parskip}
\usepackage{url}
\usepackage{hyperref}
\usepackage{bm} % Nicer bolds for vectors (including bold Greek letters and bold grad symbol)
\usepackage{verbatim} % Include files verbatim using \verbatiminput

\usepackage{graphicx}
\usepackage{rotating}
\usepackage{subfigure}
% TikZ: A package for basic line drawings:
% https://www.sharelatex.com/blog/2013/08/27/tikz-series-pt1.html
\usepackage{tikz}
\usetikzlibrary{arrows,shapes,shapes.geometric,shapes.misc}
\usepackage{pgfplots}

% commath: common maths things
% http://anorien.csc.warwick.ac.uk/mirrors/CTAN/macros/latex/contrib/commath/commath.pdf 
\usepackage{commath}

\usepackage[final]{pdfpages}
\usepackage[autostyle]{csquotes}

% Some shortcuts for vector calculus. 
\newcommand{\bs}{\boldsymbol}
\newcommand{\grad}{\bs{\nabla}}
\newcommand{\cross}{\times}
\newcommand{\curl}{\grad\cross}
\newcommand{\divg}{\grad\cdot}
\newcommand{\DDtfull}{\left(\frac{\partial}{\partial t} + \bs{u}\cdot\grad\right)}
\newcommand{\DDt} [1] {\frac{\mathrm{D}#1}{\mathrm{D}t}}
\newcommand{\dDDt} [1] {\displaystyle\frac{\mathrm{D}#1}{\mathrm{D}t}} 

% Asterisks: \ask
%%% See http://tex.stackexchange.com/questions/59183/normal-high-asterisk-in-equation-mode
%%% magic code starts
\mathcode`*=\string"8000
\begingroup
\catcode`*=\active
\xdef*{\noexpand\textup{\string*}}
\endgroup
%%% magic code ends

% Shortcuts for some mathematical functions
\newcommand{\sgn}{\mathrm{sgn}}
\newcommand{\erf}{\mathrm{erf}}
\newcommand{\erfc}{\mathrm{erfc}}
\newcommand \reals {\mathbb{R}}
\newcommand \complex {\mathbb{C}}
\newcommand \Rey {\mathrm{Re}}

%%% Column vectors: see http://tex.stackexchange.com/questions/2705/typesetting-column-vector
\newcount\colveccount
\newcommand*\colvec[1]{
        \global\colveccount#1
        \begin{pmatrix}
        \colvecnext
}
\def\colvecnext#1{
        #1
        \global\advance\colveccount-1
        \ifnum\colveccount>0
                \\
                \expandafter\colvecnext
        \else
                \end{pmatrix}
        \fi
}

\title{Unofficial Freshers' Guide}
\author{Erin Carlton (\texttt{ekc27}), George Long (\texttt{gl388}), 
    George Moore (\texttt{gam38}), Anand Patel (\texttt{ajp229})}
\date{}

\begin{document}

\maketitle

\section{Introduction}

First of all, a huge congratulations on meeting your offers and making it through A levels, interviews, STEP, UCAS, parents, teachers, open days, and all the other stress sixth-formers have to put up with! Be proud of this amazing achievement and I hope you are all looking forward to the years ahead; studying perhaps the best undergraduate Mathematics course in existence in the beautiful city of Cambridge.

Before we get into maths specific information, I wanted to give some general advice for your first year of university as a whole:

\paragraph{Be organised!} Good organisation will make everything at university much simpler. Unlike school, you won't have people constantly reminding you of deadlines or where you need to be when: it's all up to you. Get used to checking your emails 2 or 3 times a day and make sure to have some sort of planner/timetable of your lectures, supervisions and example classes.  Keep clear, well organised notes and file your example sheets as you complete them. Finally, be continually aware of when your deadlines are so that you can prioritise your workload and avoid leaving too much until the last minute. It is much easier and efficient to stay on top of your work than it is to catch up.
 
\paragraph{Make the most of the resources and opportunities.} You are paying for your university education so you might as well get as much out of it as you can - both academically and socially. In particular, don't be afraid to ask questions and seek help from supervisors and other students (but please do remember that everyone has their own work to get done and will have times that they are very stressed out/busy). 

\paragraph{Abandon perfectionism.} It is a common experience to walk into your first lecture confident in your mathematical ability and leave your lecture at the end of the week feeling like you know nothing.  The whole point of studying at Cambridge is that it's hard work, but the simple fact that you've made it through the rigorous application process to get here today guarantees that you're more than capable of succeeding. Maintain perspective, look after yourself and realise that it's impossible to master everything. Prioritise learning over getting things right: you are going to progress so much more by making mistakes rather than letting the fear of being wrong stop you from trying.
At the end of the day, your happiness and health is far more important than your achievements.

There is a lot of information in this guide, so don't worry about taking it all in at once. Keep looking back throughout first term when the tips and advice will make a bit more sense.

\section{Course descriptions}

\subsection{Michaelmas courses}

\subsubsection{Numbers and Sets}
		 
Fun and challenging, the N\&S course includes things like proof by induction, modular arithmetic, countability and series. Probably the course with the least actual `content', it is as much an introduction to higher maths in general as it is to number and set theory. That being said, the example sheets typically have some easy questions, a few tricky questions, and some downright nigh-on-impossible questions best left until the end (nearly always including, but not limited to, the last 2 or 3 questions).

Top Tip: Don't stress about the final/impossible questions. Have a go, but focus on earlier ones first, and keep in mind that the exam questions are much, much easier (and actually based on lectured material)

\subsubsection{Groups}
 				 
Interesting and beautiful, the Groups course gives an introduction to what groups are, group actions, Lagrange's Theorem, normal subgroups (amongst other things) before moving into some specific types and uses of groups. It also serves more generally as an introduction to abstract algebra, which is a large area of pure mathematics. The course also blends beautifully with Vectors and Matrices as the two approach `Matrix groups' (yes\dots they're a thing) and vector spaces from two seemingly separate, but fundamentally linked paths. Also gives a quick introduction to the Riemann Sphere, which is just the best thing.

Top Tip: Become comfortable with the basic definitions of group theory before arriving (e.g. what is a group). It'll make the first few lectures much easier, and smooth your settling in. Also, when you get to it in lectures, really try to understand the First Isomorphism Theorem.

\subsubsection{Vectors and Matrices} 		
 
Useful and fundamental, the V\&M course will probably seem the most difficult thing in the world at first because it introduces a lot of concepts which, by the end of the year, you will struggle to believe you ever did maths without (vectors, kernels, linear maps, image spaces, eigenvectors). However, as you get to grips with these, it actually becomes one of the easier courses. If you didn't do them at school, you might find it useful to revise things like 3x3 matrices and cross product so you don't feel like crying in the first few lectures.

Top Tip: Suffix notation seems horrendous, but it's so so useful. Do the suffix notation practice sheet, and try to get to grips with it ASAP once it is lectured.

\subsubsection{Differential Equations}
	 
Enjoyable and accessible from the start,  DEs is the course most like A-levels as it is highly methods based, and covers a lot of stuff that you probably know already (though it's no problem if you don't). Another of the easier courses you'll do, even if it doesn't come together to start with, it probably will by exams.  That's not to say there won't be tricky bits, especially getting to grips with PDEs and systems of linear DEs. It covers lots of different ordinary DEs, loads of different methods for solving them, and then partial DEs. 

Top Tip: Practice, practice, practice.

\subsection{Lent Courses}

\subsubsection{Vector Calculus}
		 
Really interesting course, and not too difficult once you've got the hang of suffix notation (though if you didn't understand it in V+M, this course will hit you right in the weakness!). Extending calculus methods from the familiar scalar-valued scalar functions all the way to vector valued vector functions (otherwise known as a vector field\dots like air flow in a room), it covers lots of integrating over surfaces and areas and volumes, the Integral Theorems, and tensors. Most of the `proofs' are pictures, which is refreshing (apparently, proper proofs come in later years). `Tensors' forms a quite independent sub-course for the last 6 or 7 lectures. Try not to worry too much about what they are\dots just focus on how to use them.

Top Tip: Get to grips with suffix notation before starting Lent Term.

\subsubsection{Analysis}
	 	
Ah, analysis. Probably the most marmite of the first year courses. This beautiful and fundamental course investigates and solidifies your understanding of concepts such as convergence (of sequences and series), continuity (of functions), differentiability/differentiation, power series and Riemann Integration. For example, you may think differentiation is the opposite of integration. While this is not `wrong' per se, there are many problems: Firstly, it doesn't actually mean anything. Secondly, what it implicitly means doesn't work for most functions. In this course, you will find out what the relationship between integration and differentiation really is, along with much much more. Example sheets tend to be like N\&S, with mostly relevant, doable questions, and a few nearly impossible ones designed to give you something to do when everyone else is working and you're bored.

Top Tip: Analysis is like learning a completely new language from scratch. Expect slow progress at first, and don't be disheartened by it. It gets better. Also, Sir Tim Gowers. Just listen to him.

\subsubsection{Probability}

This course starts with an axiomatic approach to probability, before moving onto discrete and continuous random variables (while visiting some particularly important distributions and types of problem along the way), and some approximation theorems. While it is generally a widely enjoyed course (I don't know anyone who would say it was their least favourite first year course), it seems that, more than most courses, the difficulty can vary significantly from person to person depending on their `affinity' for probability. However, example sheets are usually really fun and interesting.

Top Tip: Probability problems can often be made much easier by framing the question slightly differently. I would try to keep this in mind while attempting harder example sheet questions.

\subsubsection{Dynamics and Relativity}

Fascinating and fun, D\&R is a slightly marmite, highly applied course which takes a look at both general methods for solving mechanical problems (except now you'll use vectors, and thereby wonder why on Earth A-level mechanics is so stupid), along with some special dynamical problems such as Orbits, Projectiles under the Coriolis force, and much more. Also includes an essentially independent special relativity sub-course which is really interesting but can be hard to get your head around. 

Top Tip: Try actually visualising problems, and draw diagrams. In many cases, I found that doing so made solving the maths of the problem significantly easier.

\section{How does it actually work?}

So, how do you actually go about learning all the maths while you're here? 

There are four main elements to doing maths at Queens': Lectures; Example Sheets; Example Classes and Supervisions. There are other small things like the occasional meeting with your DoS or a session with us lot in the bar, but most of your maths time will be on these. I'll go over each in a bit more detail before explaining the actual day to day/week to week structure a bit.  

\subsection{Lectures}

The actual content of these will be gone over elsewhere, but here's the overall idea. Monday to Saturday from 10-12 in the morning you wander over to the Cockcroft lecture theatre (we'll take you there once) and sit down on a long bench somewhere in the hall. Aim not to be late, partly out of politeness and partly because the seats right at the sides can be pretty hard to see from. 

Lectures officially run from five past until five to, so are 50 minutes long, meaning you get a nice break in the middle. A lecture is a one way process almost all of the time, so the person standing there talks and writes, and you are quiet. That said, if you think something's wrong, definitely speak up! This is said all the time, but can take a bit of nerve to actually do. Trust me, everyone else will thank you for it. The time not to speak up is when you don't get it or get a bit lost. This will happen. Regularly. There will be lectures when you get behind and lose track and get lost, or when the moment the lecturer opens their mouth you're already lost. Do not panic. 

So what do you actually do in a lecture? The overall aim is to have your own written notes on the whole course when the lectures are over, how you achieve that is up to you. The most common practice is to listen and take notes during the lecture and this means essentially writing down everything the lecturer writes down in a pad. For some people this will be their final neat copy (me), and for others they'll go away after and copy it up neat. Also, don't feel you have to take notes if you can get away with not. I took notes diligently all year, only to realise in the last term that I didn't really listen to what was being said, I just wrote like a zombie. Sometimes printed notes are available, and then my favourite thing to do was to print these and just make notes on them as the lectures went along, so I could really listen to what was being said and absorb it better. Unfortunately I just can't tell you what you're going to like. Start by writing as the lecturer talks and experiment. It is very likely you won't settle on your perfect method for several weeks (or terms in my case), so don't panic if your first few weeks of notes are a bit of a mess. 

\subsection{Example Sheets}

Don't contain examples. This is where you actually answer questions. The bulk of your time will be spent doing these, with one for every six lectures worth of content (which works out at one per course per fortnight, or two per week). 

They can contain any number of questions of any difficulty, and can be really fun or really dull. There'll be some `bookwork' questions, where you have to apply the new maths you've been taught to relatively accessible questions, right up to fiendish, horrible questions that are there for you to do if you run out of other things (but are quite fun). Numbers and Sets is the worst for this, you probably won't hand in a completed sheet ever and if you think you have, your supervisor will tell you it's wrong. Trust me.

General advice for these:
\begin{itemize}
    \item Start them as early as possible, pin them up on your wall so you don't forget about them.
    \item Do as much as you can `blind'. This means no notes to help you, no internet, no mates, just you and a pen. This is how you really get better. Discussing tricky problems with friends is ok, but you should never let someone tell you something or give you a hint without having tried it for long enough to be sure you'd never have got it for yourself. 
    \item Email supervisors for help if you're stuck. They know how to give the best hints and help you without giving anything away. If you really can't do something, write down what you tried and why you think you're stuck. It's all good stuff for allowing your supervisor to better help you out.
\end{itemize}

\subsection{Supervisons}

And this is where Cambridge is unique (I think). This is a one hour session with you, your supervision partner and your supervisor. You tend to work through the sheet that you handed in and do general stuff about the six lectures it was on. 

Here is where you ask questions and cement your understanding. With supervisions, you'll always get out a lot as long as you're sentient because we have awesome teachers and a great system, but if you prepare for them and come knowing exactly what you don't know, you can get more out of an hour than seems plausible. What you come away with on paper will vary from supervisor to supervisor, but don't stress too much about having perfect notes from them, it's the listening that matters. General advice: don't be too tired, hungover, or late.

\section{Structure of the First Year}

First year will begin with you catching Fresher's flu and whaling in bed for 2 weeks because you can't do maths any more, but not wanting to leave because of all the free wine Cambridge has to offer. Don't panic. This is a joke...ish. Indeed, as it's been mentioned, you will begin the year feeling somewhat defeated. There is a massive learning curve, which can be off-putting, but believe me and every other person you ask, you will find a way to deal with it. You will have 2 lectures a day Monday to Saturday (I am so sorry, has anyone told you about Saturday lectures yet?) which doesn't sound like much, but trust me, it's plenty. From each lecture course you will have written about 50-80 pages of notes, which is a lot of information and there's no way you'll learn it all in term time. But don't worry, maths has a nice system.

The first two terms have a similar format. Each consists of 4 lecture courses, all 24 lectures long. In each term (it is very likely) you will do 12 examples sheets in term time and 4 in the following holiday (sorry, we say `vacation', `holiday' implies no work\dots  ha\dots ha). Then Easter (3rd) term is exam term. In this term there will be lecture courses, but they are for second year tripos (including second year computer projects) and you are not supervised on these till Michaelmas (1st term) 2nd year. So essentially Easter term is all about exams. You are not learning anything new in this time that you will be examined on any time soon. Personally, I really liked this. Maths, as I'm sure you are aware, is a subject that takes a lot of consolidation and practise. So completing first year courses nearly 3 months before exams gives you proper time to really get your head around them. From my experience, I felt the first two terms were about understanding the lecture notes and the content of each course. I used Christmas and Easter vacation to re-read lecture notes and address specific issues with courses - you'll be amazed how much more sense lecture notes make second time round. I then used Easter term to `revise' if you will. Unfortunately, much like at school/college, you do need to regurgitate a lot of information (proofs/derivations/definitions) for the exam. This isn't too difficult if you do it right, but isn't worth stressing about till 3rd term. For now, my key advice for first term is to read your lecture notes after every lecture and make sure it really makes sense to you. You won't be able to learn every proof at first, but I'd say it's important to learn the key theorems and definitions so things make a little more sense in supervisions.

The structure of exams is very neat. Four 3 hour papers with 2 courses per paper (no coursework/dissertations at all first year). The exams aren't marked like A level papers. It's far less robotic and the markers use common sense. It is ensured you'll get the grade you deserve. But don't worry about this too much. You're here to learn maths primarily so focus on the content. A quick aside though: maths exams finish a bit earlier than the others, so you'll have 2 weeks of residence at the end of the year with no work where you can enjoy Cambridge and May Week.

\section{Extra-curricular maths}

Cambs is known for its top notch extra curricular activities. Societies include:
\begin{itemize}
    \item The Archimedeans (they will probably give you Domino's during week 1)
    \item Emmy Noether society (women in maths society)
    \item Trinity Maths Society (home of free port)
    \item QMS (Queens' Maths Society\dots  the best one (home of free cake))
\end{itemize}
There are of course non-mathematical societies\dots but who cares about them?!

As well as societies, there's also extra lectures!!!! You heard right!! How exciting!!?!!

These are pretty great though. First term there will be introduction to mechanics lectures (for those who did not do M3 or the equivalent at A level). These are non-compulsory but really should be attended if you require them. As well as these, there's a lecture series during first term on Ancient Mathematics. From what I've heard these are top banter and a nice relief from the intensity of tripos. In third term, there will also be non-examinable lectures on concepts in theoretical physics. These are of great interest to people looking to go down the applied route. There are no examples sheets for these courses. Just fun. Extended fun.

\end{document}

